\documentclass{article}
\parindent 0pt
\title{Replication Script to JSS Submission}
\usepackage{Sweave}
\begin{document}

\author{Lindsay Rutter, Susan VanderPlas, Dianne Cook, Michelle A. Graham}
\date{\today}
\maketitle
\Sconcordance{concordance:replicationScript.tex:replicationScript.Rnw:%
1 3 1 1 0 14 1 1 2 1 0 3 1 11 0 1 2 2 1 1 2 1 0 1 1 5 0 1 1 6 0 1 2 3 1}

\tableofcontents

%%%%%%%%%%%%%%%
%% Section 2 %%
%%%%%%%%%%%%%%%
\section*{2. Available software}

The two figures in this section of the paper were not produced using source code from our ggenealogy package. Instead, they were produced from external software for demonstration purposes.

%%%%%%%%%%%%%%%
%% Section 3 %%
%%%%%%%%%%%%%%%
\section*{3. Example datasets}

First, we can load and examine the structure of the soybean genealogy example dataset (called sbGeneal):

\begin{Schunk}
\begin{Sinput}
> rm(list=)
> library("ggenealogy")
> data("sbGeneal")
> str(sbGeneal)
\end{Sinput}
\begin{Soutput}
'data.frame':	390 obs. of  5 variables:
 $ child       : chr  "5601T" "Adams" "A.K." "A.K. (Harrow)" ...
 $ year        : num  1981 1948 1910 1912 1968 ...
 $ yield       : int  NA 2734 NA 2665 NA 2981 2887 2817 NA NA ...
 $ year.imputed: logi  TRUE FALSE TRUE FALSE FALSE FALSE ...
 $ parent      : chr  "Hutcheson" "Dunfield" NA "A.K." ...
\end{Soutput}
\end{Schunk}

After that, we can load and examine the structure of the academic statistician genealogy example dataset (called statGeneal):

\begin{Schunk}
\begin{Sinput}
> data("statGeneal")
> dim(statGeneal)
\end{Sinput}
\begin{Soutput}
[1] 8165    6
\end{Soutput}
\begin{Sinput}
> colnames(statGeneal)
\end{Sinput}
\begin{Soutput}
[1] "child"   "parent"  "year"    "country" "school"  "thesis" 
\end{Soutput}
\end{Schunk}

Convert sbGeneal into igraph object:

\end{document}
